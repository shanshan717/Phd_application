\documentclass[11pt,a4paper]{article}
\usepackage{geometry}
\geometry{margin=1in}
\usepackage{enumitem}
\usepackage{hyperref}
\hypersetup{
    colorlinks=true,
    linkcolor=black,
    urlcolor=blue
}
\usepackage{fancyhdr}
\usepackage{parskip}
\setlength{\parskip}{0.5em}
\setlength{\parindent}{0em}

% Set up header for Last Updated
\pagestyle{fancy}
\fancyhf{}
\rhead{Last Updated: April 27, 2025}

\begin{document}

\begin{center}
    {\LARGE \textbf{Shanshan Zhu}} \\
    \vspace{0.2cm}
    \href{mailto:zhushanshan0717@gmail.com}{zhushanshan0717@gmail.com} \quad |\quad \href{https://github.com/shanshan717}{github.com/shanshan717} \\
\end{center}

\hrule
\vspace{0.5cm}

\section*{Education}

\textbf{Nanjing Normal University, Nanjing, China (GPA: 3.71/4.00)} \hfill Sep 2023 -- Jun 2026 (Expected) \\
M.Sc. in Basic Psychology \\
Relevant Coursework: Bayesian Data Analysis in Psychology, Experimental Psychology, R Language and Its Applications in Psychological Research

\vspace{0.3cm}

\textbf{Hangzhou Normal University, Hangzhou, China (GPA: 3.80/5.00)} \hfill Sep 2019 -- Jun 2023 \\
B.A. in Health Services and Management

\vspace{0.5cm}

\section*{Research Experience}

\textbf{Principal Investigator, Self-Reference Database}, \href{https://doi.org/10.57760/sciencedb.j00001.00469}{[web]}  \hfill Jan 2023 -- Present \\
- Led the construction and data cleaning of a self-reference dataset shared on the Science Data Bank platform.

\textbf{Principal Investigator, ALE Meta-Analysis} \hfill Feb 2024 -- Present \\
- Leading an ALE meta-analysis project using Python (NiMARE package) to synthesize neuroimaging findings on self-referential processing. \\
- Project title: \textit{``Abnormal Brain Activations During Self-Referential Processing Across Psychiatric Disorders: A Transdiagnostic Neuroimaging Meta-Analysis''} (In Progress).

\textbf{Research Assistant, Legal Education and Brain Function Study} \hfill Feb 2024 -- Present \\
- Co-authored a manuscript and conducted fMRI preprocessing and analysis to investigate the effects of legal education on cognitive functions.

\vspace{0.5cm}

\section*{Teaching Experience}

\textbf{Teaching Assistant, Bayesian Methods in Psychology} \hfill Sep 2024 -- Jan 2025 \\
Nanjing Normal University | Instructor: Prof. Chuan-Peng Hu \\
- Led laboratory sessions and tutorials, supporting students with R and Python-based Bayesian statistics.

\vspace{0.5cm}

\section*{Professional Training}

\textbf{FSL Neuroimaging Analysis Workshop} \hfill Jul 2024 \\
Institute of Psychology, Chinese Academy of Sciences \\
- Completed a one-week intensive training course and received certification.

\textbf{MRI Technician, CCNP Project, Nanjing Site} \hfill Oct 2024 -- Jul 2025 \\
- Conducted MRI scans of school-aged children and contributed to longitudinal neuroimaging data quality control.

\vspace{0.5cm}

\section*{Internship Experience}

\textbf{AI Trainer Intern, NetEase Intelligence} \hfill May 2023 -- Sep 2023 \\
- Annotated NLP training datasets and contributed to cognitive-linguistic tasks to improve model performance.

\textbf{Research Intern, Sir Run Run Shaw Hospital} \hfill Jul 2022 -- Dec 2022 \\
- Assisted with research administration, documentation, and ethical review processes.

\textbf{Medical Observer, Affiliated Hospital of Hangzhou Normal University} \hfill Jul 2021 -- Sep 2021 \\
- Gained early clinical exposure in psychiatry and behavioral health units.

\vspace{0.5cm}

\section*{Conferences and Presentations}

\textbf{24th National Congress of Psychology}, Chengdu, China \hfill Oct 2023 \\
- Oral Presentation: \textit{``Abnormal Brain Activity in Self-Referential Processing Among Psychiatric Patients: An ALE Meta-Analysis''}

\textbf{Frontiers in Cognitive Neuroscience Hackathon}, Hangzhou, China \hfill Aug 2024 \\
- Initiator and Team Lead: \textit{``Python-Based fMRI Data Analysis Pipeline''}

\vspace{0.5cm}

\section*{Academic Service}

\textbf{Chinese Open Science Network} \hfill 2023 -- Present \\
- Academic Secretary: Supported the planning and organization of scholarly activities. \\
- WeChat Content Editor: Produced, formatted, and published official articles to promote open science initiatives.

\vspace{0.5cm}

\section*{Skills}

- \textbf{Programming}: Python, R, MATLAB \\
- \textbf{Neuroimaging}: FSL (certified), fMRI preprocessing, ALE meta-analysis \\
- \textbf{Tools}: Microsoft Office, LaTeX, GitHub

\end{document}
